\documentclass{article}
\usepackage{graphicx}
\usepackage{amsfont,amsmath,cite}
\title{Solving small shogi}
\author{I. van Duijn \\ Utrecht University}
\begin{document}
\maketitle

\section{Abstract}
%TODO

\section{Introduction}
Even since before
%TODO
%Research area of solving games.
%The game of shogi and relevant topics.
%Various methods for solving, PN is chosen.

\section{Method}
We are dealing with generic small sized shogi variants, so representation and algorithms employed have to be sufficiently generic as well.
A small shogi variant is defined by the size of the board, the initial configuration of black (white's is $180\deg$ rotationally symmetrical) and
the size of the promotion zone. The initial configuration consists of a set of pieces, each of which has one or two move sets (taking into account
promotion). Any algorithm that aims to prove the game theoretical value of a variant then only has to have access to two functions based on this
definition: a move generator which can map positions to a set of successing positions, and an evaluation function which recognises terminal
positions and computes their value.

\subsection{Representation}
As state before, we have to be able to generically represent board positions. A well known technique in computer chess, bitboards, has been
shown ~\cite{grimbergen2007using} to be fairly effective in shogi as well. A bitboard is a bit pattern of (at least) the size of the board,
in which each bit encodes some information (e.g. if it is occupied) about the corresponding square. Since the size of the bit pattern can
easily be increased to accomodate larger boards, they are well suited for our purpose. Also, as will be explained later, they allow for easy
generic move generation.
%TODO: Briefly explain implementation of boards.
%TODO: Move generation and evaluation function.


%subsection: Proof Number search

%subsection: Breadth first search

\section{Result}
%subsection: PN failed

%subsection: Combinatorial splosions.

\section{Discussion}

\end{document}
