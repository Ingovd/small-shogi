\documentclass{paper}
\usepackage{graphicx}
\usepackage{amsfonts}\usepackage{amsmath}
\title{Winning strategies for small shogi boards}
\author{I. van Duijn}
\begin{document}
\maketitle
\section{Introduction}
Shogi (Japanese chess) is a game similar to western chess. It is played on a $9 \times 9$ board and every
player has 20 pieces, some of which correspond to western pieces and others are slightly different.
From a chess programmer point of view, the main difference between shogi and chess is the so called drop rule.
A piece that is captured by a player now belongs to him and can be placed on \textit{any} vacant square in one of his subsequent turns.
This gives shogi a game tree complexity of $10^{226}$, compared to $10^{123}$ of chess.

\section{Computer shogi}
Computer shogi is an active field of research. The main interest there is how we can program computers 
such that their play peers with -- and eventually even surpasses -- human professional players. Shogi, being complexer
than western chess, has a game tree too large to make exhaustive search even near feasible. Consequently, research
focusses on clever ways to search the game tree in order to find a good heuristic.

Unlike in western chess, end game situations
in shogi are not significantly less complex than mid game situations. In fact, any board configuration resulting from the standard
starting setup has exactly 40 pieces in play. Because of this there is no end game database for shogi with end games completely
solved. However in shogi so called mating problems, which resemble end game situations, are extensively studied. There exist several
programs that can completely solve a lot of these problems, some of which are quite large. Unfortunately solutions to these shogi mating problems
are not the equivalent of solutions to western chess end games; in shogi, such a solution cannot be directly applied to find a winning move.

\section{Mating problems}
In a mating problem, the attacker is only allowed to make checking moves, while the defender prolongs mate as long as possible.
Usually not all the pieces are in play, specifically the attacker's king is always absent. A mating problem should have a unique solution and
is considered invalid if it has multiple solutions.

The absence of one king already shows that a mating problem differs greatly from a real end game situation. Also, not all pieces are in play and
it is possible that a winning move in normal end game play is not a checking move. However the fact that a solution to a mating problem resembles a winning
strategy is interesting.

\section{Solving small shogi boards}
It is not known whether there is a winning strategy for shogi since drawing is possible. If a winning strategy exists though,
it is very interesting to know what moves are good (winning) and which are bad. It was already evident that solving normal shogi
is infeasible, but solving small boards with fewer pieces might prove interesting. It is probably feasible to completely solve some of these
small boards and thus answer questions like is there a winning strategy, if so for which player and which of his moves are good.

Subsequently, what might be interesting is how altering the rules of a small board -- like changing the pieces, drawing rule or increasing the size of the board --
influences winning moves. Of course, solving small boards should first prove to be feasible.

\section{What to do}
From a small project point of view, studying how we can solve small shogi boards probably comprises the following:
\begin{itemize}
	\item Small literature study, analyse which methods might be useful for solving small shogi
	\item Figure out methods to solve mating problems
	\item Implement a combination of methods and try to solve small shogi board
	\item Analyse results
\end{itemize}
A possible obstacle is that a lot of literature on this subject is in Japanese and not readily available. Fortunately there is
also a decent amount of readily available research papers written in English, which in a lot of cases heavily cite Japanese research.
This at least means that available research is based on Japanese papers.
\end{document}